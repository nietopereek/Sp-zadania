\documentclass[12pt,a4paper]{article}

% ustawienia marginesu
\usepackage[left=1.6in,right=.8in,top=1.5in,bottom=1.5in]{geometry}

% polskie reguły dzielenia wyrazów itd
\usepackage{polski}

% polskie znaki zakodowane w UTF8
\usepackage[utf8]{inputenc}

% automatyczne generowanie odnośników w plikach PDF
%\usepackage[pdftex,linkbordercolor={0 0.9 1}]{hyperref}

% pakiety matematyczne
\usepackage{amsthm,amsmath,amsfonts,amssymb,mathrsfs}

% ładne składanie odnośników do stron www
\usepackage{url}
\usepackage{textcomp}
% rozbudowane możliwości wypunktowań
\usepackage{enumerate}

% możliwość dodawania plików graficznych
\usepackage{graphicx} 

%%% definicje twierdzeń itd :)
\newtheorem{tw}{Twierdzenie}[section]
\newtheorem{stw}[tw]{Stwierdzenie}
\newtheorem{fakt}[tw]{Fakt}
\newtheorem{lemat}[tw]{Lemat}

\theoremstyle{definition}
\newtheorem{df}[tw]{Definicja}
\newtheorem{ex}[tw]{Przykład}
\newtheorem{uw}[tw]{Uwaga}
\newtheorem{wn}[tw]{Wniosek}
\newtheorem{zad}{Zadanie}

% oznaczenia zbiorów liczbowych
\DeclareMathOperator{\R}{\mathbb{R}}
\DeclareMathOperator{\Z}{\mathbb{Z}}
\DeclareMathOperator{\N}{\mathbb{N}}
\DeclareMathOperator{\Q}{\mathbb{Q}}


% wartość bezwzględna, norma, iloczyn skalarny, nośnik, rozpięcie przestrzeni...
\providecommand{\abs}[1]{\left\lvert#1\right\rvert}
\providecommand{\var}[1]{\operatorname{var}(#1)}

% fajne nagłówki i stopki na stronie
\usepackage{fancyhdr}
\pagestyle{fancy}
\fancyhf{}
\fancyfoot[R]{\textbf{\thepage}}
\fancyhead[L]{\small\sffamily \nouppercase{\leftmark}}
\renewcommand{\headrulewidth}{0.4pt} 
\renewcommand{\footrulewidth}{0.4pt}

% typowe dane dokumentu
\title{Funkcje ciągłe i różniczkowalne}
\date{30 Listopada 2010}

% tu podaj swoje imię i nazwisko!
\author{Witold Bołt}

% zaczynamy dokument
\begin{document}
 
% pokaż tytuł
\maketitle

% spis treści
\tableofcontents

\section{Funkcje ciągłe}

\begin{df}
(funkcja ciągła). Niech f: 
\begin{math} (a,b) \to R \ oraz \ niech \ x_{0} \in (a,b).
\end{math}
Mówimy, że funkcja f jest ciągła w punkcie $x_{0}$ wtedy i tylko wtedy, gdy:
\center $ \bigvee_{\epsilon>0}\exists_{\delta>0}  \bigvee_x \in (a,b) |x-x_{0}| < \delta\rightarrow |f(x)-f(x_{0}| < \epsilon  $ 
\end{df}
\begin{ex}
Wielomiany, funkcje trygonometryczne, wykładnicze, logarytmicz-
ne są ciągłe w każdym punkcie swojej dziedziny.
\end{ex}

\begin{ex}
Funkcja f dana jest wzorem:

	
\begin{center}
	 $f(x) =  \begin{cases} {1} x+1 \ dla  \ x \neq 0 
		\\  0 \ dla \ x = 0  \end{cases}$	
\end{center}	
	\end {ex} 

Jest ciągła w każdym punkcie poza $x_{0} = 0.$

Niech $\mathbb{Q}$ oznacza zbiór wszystkich liczb wymiernych.

\begin{ex}
Funkcja f dana wzorem:

f(x) = $ \left \lbrace  0 \ dla \ x \in \mathbb{Q}   \\
1 \ dla \ x \not \in \mathbb{Q}
 \right \rbrace $ \\
nie jest ciągła w żadnym punkcie.
\end{ex}

\begin{ex}
Funkcja f dana wzorem: \\nie 
f(x) = $ \left \lbrace 0 \ dla \ x \in \mathbb{Q} \\ 
	x \ dla \  x \not \in \mathbb{Q} \right \rbrace $ \\
Jest ciągła w punkcie $ x_{0} = 0, $ ale nie jest ciągła w pozostałych punktach dziedziny.
\end{ex}

\begin{zad}
Udowodnij prawdziwosć podanych przykładów.
\end{zad}

\begin{df}
Jesli funkcja f: A $ \to \mathbb{R} $ jest ciągła w każdym punkcie swojej dziedziny A to mówimy krótko,że jest ciągła. \\ \\
Poniższe twierdzenie zbiera podstawowe własnosci zbioru funkcji ciągłych.
\end{df}

\begin{tw}
Niech funkcje f,g: R $\in \mathbb{R}$ będą ciągłe, oraz niech $ \alpha,\beta \in \mathbb{R}.$ \\
Wtedy funkcje: \\ \\
a) $h_{1}(x) = \alpha \cdot f(x) + \beta \cdot g(x),$
b) $h_{2}(x) = f(x) \cdot g(x),$
c) $h_{3}(x) = \frac {f(x)} {g(x)} $ ( o ile g(x) $ \neq 0 \ dla \ dowolnego \ x \in \mathbb{R} $),
d) $h_{4}(x) = f(g(x)),$ \\ \\
są ciągłe.
\end{tw}
Następne twierdzenie zwane powszechnie „własnością Darboux” lub twierdze-
niem o wartości pośredniej ma liczne praktyczne zastosowania. Mówi ono o tym,
że jeśli funkcja ciągła przyjmuje jakieś dwie wartości, to przy odpowiednich założe-
niach co do dziedziny, przyjmuje też wszystkie wartości pośrednie. Możemy sobie to
łatwo wyobrazić na przykładzie funkcji, która opisuje zmianę temperatury w czasie.
Jeśli o 7:00 było  -1 \textcelsius \
 a o 9:00 było 2 \textcelsius \
, to zapewne gdzieś między 7:00 a 9:00 był
taki moment, że temperatura wynosiła dokładnie 0 \textcelsius



\begin{tw}
Niech f: [a,b] $ \to \mathbb{Q} $ ciągła, oraz niech f(a) $\neq$ f(b). Wtedy dla dowolnego $ y_{0} \in conv \lbrace f(a), f(b) \rbrace \ istnieje \ x_{0} \in [a,b]\ takie, że \ f(x_{0}) = y_{0}. $
\end{tw}

\section{Różniczkowalnosć}
\begin{df}
Niech g: (a,b) $ \in \mathbb{R} \ , x_{0} \ in (a,b) $ oraz f ciągła w otoczeniu punktu $ x_{0}. $ Jesli istnieje granica: 
\center
$ {\mathop{\lim}\limits_{x \to x_{0}}  \frac {f(x) - f(x_{0})} {x-x_{0}}} $ \\
i jest skończona, to oznaczamy ją przez $ f^{\prime}(x_{0}) $ i nazywamy pochodną funkcji f w punkcie $ x_{0} $
\end{df}

\begin{df}
Jeśli funkcja f posiada pochodną w każdym punkcie swojej dziedzi-
ny, to mówimy, że f jest różniczkowalna. Istnieje wtedy funkcja f0
, która każdemu
punktowi z dziedziny funkcji f przyporządkowuje wartość pochodnej pochodnej
funkcji f w tym punkcie.
\end {df}

\begin{ex}
Wielomiany, funkcje trygonometryczne, wykładnicze, logarytmicz-
ne są różniczkowalne w każdym punkcie dziedziny.
\end{ex} 

\begin{ex}
Funkcja f(x) = |x| jest ciągła, ale nie posiada pochodnej w punkcie 
$x_{0} = 0 $
\end{ex}

\begin{tw}
Niech f: [a,b] $ \rightarrow \mathbb{R} $ ciągła i różniczkowalna na (a,b). Dodatkowo
niech $ f \prime (x) \neq 0 \ dla \ x \in (a,b), \ oraz \ niech \ m = min_{x \in [a,b]} f(x), M = max_{x \in [a,b]} f(x). $ \\
wtedy napewno f(a) = m, f(b) = M lub f(a) = M i f(b) = m.
\end{tw}






\end{document}